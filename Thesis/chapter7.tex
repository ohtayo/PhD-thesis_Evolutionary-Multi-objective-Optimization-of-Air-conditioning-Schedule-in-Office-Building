\chapter{結論}
\label{chap::conclusion}
\section{得られた知見}

\hspace{1zw}

本研究では,オフィスビルの空調設定スケジュールの多目的最適化について,より実用的な解を探索するアプローチについて複数検討した.具体的には,まずオフィスビルのテナントを想定した1部屋の室内快適性およびエネルギー消費量について,数理モデルにより目的関数を定式化し,進化計算により多目的最適化を行うコンセプトを提案した.次に,オフィスビル全体の快適性およびエネルギー消費量について,EnergyPlusビルエネルギーシミュレータを使ったシミュレーションにより解評価する手法を提案した.シミュレータを用いることで大規模・複雑な問題もブラックボックスとして取り扱い進化計算を適用できる.多目的最適化手法としてPSOを多目的最適化に拡張したOMOPSOを用いて,最適化を行った.数値実験では,多目的最適化手法によって獲得されたスケジュールが制約条件の範囲内で動的に変動し,従来の設定温度一定とされた解と比較して良好な目的関数値を有する解を探索可能であった.この結果から,進化計算によるオフィスビルの空調設定スケジュール多目的最適化の実現可能性が示された.さらに,4章において,制約条件を考慮しない探索手法,単一目的最適化手法,OMOPSO以外の多目的最適化手法との比較を行った.加えて,OMOPSOにおけるアルゴリズムの各構成要素の最適化における貢献度を分析した.この結果として,提案システムで採用したOMOPSOは,室内快適性とエネルギー消費量のトレードオフを広域に近似可能で,良好な空調設定スケジュールの候補を複数提示可能なアルゴリズムであることを明らかにした.

4章で提案したシミュレータに基づく空調設定スケジュールの多目的最適化では,実用に足るスケジュールを探索可能であることを示した.一方で,解である空調設定スケジュールの評価に計算コストの高いシミュレータを用いており,最適化に要する時間が長くなってしまっていた.空調設定を適用する時間よりさらに前からしか最適化できない場合,直近の高精度な気象予報が利用できず,シミュレータのシミュレーション結果に気象予報誤差の影響が現れてしまう.また,最適化に時間がかかると,評価回数を多くすることができず,十分な探索が行われない恐れがある.そこで,5章および6章では,4章で提案した手法について,より現実性を高めるためのアプローチとして,ロバスト最適化とサロゲート最適化という二つのアプローチから改善を試みた.5章では,気象予報誤差に対してロバストな空調設定温度スケジュールの進化的多目的最適化手法を提案した.外気温予報誤差による目的関数値への影響をロバスト性指標値として定式化し,そのロバスト性指標値も目的関数に加えて多目的最適化を行うことで,気象予報誤差に対してロバストな空調設定温度スケジュールが獲得できることを示した.提案手法はロバスト性指標も目的関数として探索を行っているため,提案手法によって獲得されたスケジュールは,ロバスト性を多少犠牲にしてでも目的関数値の良いスケジュールや,目的関数値は多少悪化するがロバスト性の高いスケジュールなど多様な候補を含む.これらのスケジュールをビル管理者に提示することで,ビル管理者によるより柔軟な空調システムの運用が実現可能である.
6章では,4章で提案したシミュレーション結果に基づく空調設定スケジュールの多目的最適化手法の高速化を行うため,時系列予測を行うLSTMに基づくサロゲート評価器を用いたサロゲート最適化システムを提案した.LSTMで構成したサロゲート評価器をあらかじめ生成したシミュレータの入出力データを用いて学習させたところ,シミュレータの入出力を高精度で模擬することが可能となった.学習済みのサロゲート評価器をシミュレータの代替として多目的最適化システムに組み込み解探索を行った結果,シミュレータに基づく多目的最適化で得られる解に準じた実用的な空調設定スケジュールを,短い時間で獲得することが可能となることが明らかになった.しかし,目的関数空間および設計変数空間において訓練データが存在しない領域を探索する場合はサロゲート評価器による予測精度が低下することが明らかになった.そこで,あらかじめ用意する学習データには広範囲の領域の解を含ませること,学習データに含まれない領域の解が探索された場合に,探索された解をサンプルに追加してサロゲート評価器を再学習することで,サロゲート評価器の誤差低減による解探索性能向上が可能となることが示唆された.また,OMOPSOの拡張として,解の目的関数空間における位置に基づいて解の変異方向を決定するDOMOPSOを提案した.DOMOPSOはOMOPSOや他の多目的最適化手法よりも少ない評価回数で良好なスケジュール集合を獲得できるアルゴリズムであることを示した.

\section{今後の課題と展望}
\subsection{短期的な課題}
本研究の短期的な課題として,4章で述べたような$\epsilon$制約法などによる目的関数削減した場合との比較,サロゲートモデルの構成やパラメータ,提案した進化計算アルゴリズムの性能検証の3つについて以下に述べる.

\subsubsection{目的関数を削減した場合との比較}
4章で,多目的最適化による結果と,$\epsilon$制約法を用いた単一目的最適化による結果の比較を行った.ここでは単一目的最適化手法にPSOとDEの2手法のみ用いて比較したが,パラメータについて検討の余地がある.さらに,これらは最良の単一目的最適化手法として選定されたわけではないため,例えば先行研究\cite{Alhaider15, Xu13}などで用いられたソルバや,CMA-ESなどの他の進化計算手法を比較対象に加えて,それら既存法による単一目的最適化よりも多目的最適化のほうが良い性能が得られるかを確認する必要がある.
また,5章のロバスト最適化において,ロバスト性を多少犠牲にしてもより目的関数値が良い解を探索できることを期待して,ロバスト性を目的関数として取り扱い,4目的最適化問題とした.これらのロバスト性に関する目的関数は4章と同様に$\epsilon$制約法によって制約として取り扱うことも可能である.実際にロバスト性を制約として2目的の最適化を行った場合に,ロバスト性を目的関数として扱った4目的最適化との性能比較をする必要がある.

\subsubsection{サロゲートモデル}
6章で述べたように,LSTMを用いたサロゲート評価器は,訓練データが存在しない領域では予測精度が低下する.この問題を解決するために,探索途中の解をシミュレータで評価して学習データのサンプル数を増やしLSTMを再学習することで,探索精度向上する手法の適用を検討する.追加するサンプルの偏りや再学習の頻度によっては過学習を引き起こす恐れがあるので,再学習する際の適切な学習周期や学習パラメータを確認する必要がある.
また,6章で述べたサロゲート最適化では,サロゲートモデルはシミュレータの入出力を模擬する構成とした.そのため,外気温度,外気湿度および設計変数である空調設定スケジュールの3つの時系列データを入力とし,室内快適性およびエネルギー消費量の2つの時系列データを出力とした.そして,時系列データの学習に向くLSTMをサロゲートモデルとして採用した.しかしながら,サロゲートモデルの構成はLSTMの1つだけではなく以下のように多様な構成が考えられる.これらの構成を採用したサロゲートモデルと,6章で構築したLSTMに基づくサロゲートモデルについて,予測精度および計算時間を比較し,本問題に対して適切な構成を検討する.
\begin{itemize}
    \item 各時刻の値をそれぞれ入力・出力ユニットに持つシーケンシャルなNN
    \item 時系列データではなく目的関数値を出力とするNN
    \item 快適性とエネルギー消費量それぞれを独立したネットワークで予測するNN
    \item OnlineSVRなど追加学習が容易な回帰予測器
    \item クリギング手法などの従来手法によるサロゲートモデル
\end{itemize}

\subsubsection{進化計算アルゴリズムの改善と性能検証}

6章では,LSTMに基づくサロゲート評価器を用いた2目的の多目的最適化問題に対して,OMOPSOアルゴリズムの改善手法であるDOMOPSOを提案した.DOMOPSOが従来のOMOPSOに比較して,空調設定スケジュール最適化問題に対して良好なパレート解を探索でき,有効な手法であることを示した.一方,他の最適化問題において本手法のgBest選択方法が探索に与える影響が明らかになっておらず,今後,アルゴリズム研究に用いられるベンチマーク最適化問題群を用いて検証する.加えて,4目的の空調設定スケジュール最適化問題や空調設定スケジュール最適化以外の問題に対する有効性が検証されていない.そこで,DOMOPSOの挙動に関する解析をすすめるとともに,本論文で対象とした問題以外の多目的・多数目的最適化問題に対する有効性を検証する必要がある.

また,今回はOMOPSOアルゴリズムにおけるグローバルベストの選択方法に着目として改善手法を検討したが,同様にOMOPSOのパーソナルベストの保持・利用方法に関する工夫や,設計変数空間で他の領域を探索する工夫,解分布の均一性を高める工夫の取り込みなどの改善により,さらなる探索性能の向上を検討する.加えて,本論文で取り上げた空調設定スケジュール最適化問題などの実問題においては,解評価に時間がかかるためパラメータの試行錯誤が困難である場合が多い.そこで,パラメータの調整が不要なアルゴリズムや,問題に合わせてパラメータを適応的に変化させる仕組みを持つアルゴリズムの研究も必要であると考える.

\begin{comment}
さらなるアルゴリズム改良のため,OMOPSOおよび提案したアルゴリズムが他のNSGA-IIIなどと比較して十分良好な結果が得られた理由を調査する必要がある.例えば,NSGA-IIIなどで採用されるアーカイブにはアーカイブサイズの制限があるが,OMOPSOではサイズ制限なく非劣解をすべて格納するアーカイブを持つ.この点が良好なHV値の獲得に有効に働いていると考えられるため,アーカイブサイズの効果を検証する必要がある.
\end{comment}

\subsection{長期的な展望}
本研究に関して,実用性と現場適用を考慮した長期的な展望として,以下3つの課題を挙げる.

\subsubsection{問題の拡張}
本研究では,最適化対象として空調設備の設定のうち温度設定にフォーカスを当て,その1日のスケジュールを最適化した.しかしながら,空調設備に関する設定には,温度のほかにも風量,風向,湿度など多数の設定が存在する.加えて,オフィスビルには空調設備以外にも照明設備やポンプ・ファンなどの搬送設備,換気設備など多様な設備が存在している.それらすべての設定や運用計画を変更してビル内の快適性や機能を維持しつつエネルギー消費を抑える必要がある.
また,本研究では1日のスケジュールを対象としたが,ビルのエネルギー管理では年間のエネルギー消費量やCO2排出量を目標とすることから,年間のスケジュールまで期間を拡大してスケジューリングする必要がある.
さらに,本研究では,ビル全体に同一の設定温度スケジュールを適用した.しかし,実際にはオフィスビルの多くはテナントビルであり,フロアや部屋単位で異なるテナントが入居していることから,専有部は別々の設定を適用する必要がある.そのため,設計変数の対象をビル全体で単一の温度設定スケジュールではなく,各部屋・設備でそれぞれ固有の設定とし最適化を行うシステムについて検討する必要がある.しかし,大規模ビルに対してこのように設計変数をとると,設計変数の数が1000を超える大規模最適化問題となる.このような問題に対しても有効なアルゴリズムや,設計変数の数を低減させる次元削減方法が必要である.

\red{本研究で提案した手法を実際のビルに適用する際には,(1)ビルのシミュレーションモデルの作成,(2)シミュレーション実施による学習データセットの作成とサロゲートモデルの学習,(3)サロゲートモデルを用いた最適化の実施・実運用,の手順が考えられる.この各手順で課題を解決する必要がある.(1)のモデル作成時には,新築・既設いずれのビルにおいても,ビルのシミュレーションに直接使用可能なデータが存在することはまれであり,既存の図面や現地の機器情報からシミュレーションモデルを作成しなければならない.図面・現地機器情報からシミュレーションに必要な情報を抽出したり,図面などをもとに本論文で述べた建築用のビルモデルを構築する方法が必要である.(2)の学習時には大量のデータ生成が必要であることから,シミュレーションを多数回繰り返す必要がある.そこで,このデータ生成と学習コストを削減するため,同様の規模・設備を持つ他のビルのデータセットを用いた事前学習や,他のビルのデータで学習した結果を転移学習するなどの方法が必要である.(3)で実際に運用した結果,サロゲートモデルやシミュレータ自体に誤差が存在し,最適な運用とならない場合がある.そのため,実際のビルの運用データとシミュレータの結果を比較してシミュレータのパラメータを補正する方法や,そもそもシミュレータで考慮しきれなかった要素に対してその考慮を行う方法を用いてシミュレータを実際のビルに近づけ,サロゲートモデルを学習し直す必要がある.
}

今回は空調設備を対象としたため,エネルギー消費量と室内快適性を目的関数とした.しかしながら,ビル設備に対しては,より多様な要望が存在する.たとえば,ビルオーナーは運用コストを削減したい,オフィスビルのテナントは電気料金削減をしたい,ビル利用者は室内CO2濃度や照度を適切に保ってほしい,といった要望を持つ.\red{今回はオフィスビルを対象としたが,対象ビルの用途が異なる場合,例えば病院なら就寝している患者や立位で仕事をする労務者などオフィスとは異なるステークホルダーが数多く存在し,さらに異なる要望が挙げられる.このような異なる要望}それぞれに対して目的関数を定義し,目的関数を良くするように空調設備,照明設備,換気設備など複数の設備の設定・運用計画を最適化する必要がある.このとき目的関数の数がより多数となるが,その場合でも良好な解を探索できる多数目的最適化アルゴリズムや,全体最適を成立させたまま問題を分割する手法の開発も必要となる.

\subsubsection{ロバスト性の強化}
5章では,外気温の予報誤差にロバストな解を探索する手法を提案した.しかしながら,気象予報にはほかにも天候(晴れ,雨等)や日射量,外気湿度,雨量などがある.さらに,オフィスワーカーの部屋利用率なども想定とは異なる場合があり,気象予報同様にビルの熱負荷予測と室内快適性,エネルギー消費量計算に影響している.そこで,気温以外の気象予報情報の変動も含めたシミュレーションや,オフィスのスケジューラ情報による部屋使用率予測情報とその変動を含めたシミュレーションを行う等の手法によって,これらの影響に対してもロバストな空調設定スケジュールを獲得する手法を検討する.
また,実際に得られたスケジュールを適用する場合,予報誤差があると,空調設定スケジュールを当初予定から変更することで同様の快適性・エネルギー消費量を達成するように運用することが想定される.スケジュールの変更は,予報が外れた場合だけでなく新たな予報が発表された場合など1日の中で複数回発生する.このような運用においては,空調設定スケジュール変更時刻毎に,その日の終わりまでの目的関数値を算出して,スケジュール変更値を最適化することを繰り返す動的最適化を導入することを検討する.昨今では気象予報の精度が年々向上しており,5分といった間隔で気温予報が更新可能となっていることから\cite{Weather17},5分毎に最適化を完了してスケジュール変更する動的最適化の実現が望まれる.本論文で提案したサロゲート最適化を用いても最適化に30分かかっていることから,5分毎の動的最適化の実現には,アルゴリズムによる解探索性能向上,サロゲートモデルの性能向上,前回最適化結果の活用などによる,さらなる最適化時間の短縮が必要である.


\subsubsection{意思決定者の意思決定支援}
本研究では,\secref{sec::decision_making}で提案手法により空調設備スケジュールの非劣解集合を獲得した後,ビル管理者がビルの運用状況に合わせて最終的に1つの解を選択し,その解に基づいて空調を操作するプロセスについて述べた.しかしながら,ビル全体の場合は,ビル全体の快適性だけでなく各部屋の快適性の推移を確認したり,法規制など別の観点から妥当性を検討して選択する必要がある.さらに,\chapref{chap::robust}のように多数目的最適化の結果になると考慮する評価軸と比較対象の解の個数が増え,適切な解の選択が容易にはできなくなる.
そこで,解の妥当性検討や取捨選択を容易にし,ビル管理者の意思決定を支援する手法の検討が必要である.たとえば,並行座標プロットやレーダーチャートなど複数の可視化手段で獲得した解を可視化したり,「目的関数の評価値が近い」「スケジュールの特徴が類似」「不快申告のあった部屋の快適性が高い」といった観点や自己組織化マップ(Self Organizing Map, SOM)を使ったクラスタリングにより解を分類し抽出するなどの手法を検討する.
また,意思決定にあたっては,一度の意思決定プロセスでビル管理者の要望がすべて反映できるとは限らず,考慮しきれない要望が残ることも考えられる.その対応として,対話型多目的最適化の検討も必要であると考える.
\linebreak
\par
このように,今後,より実用性の向上と現場適用を考慮した最適化の研究に取り組むことで,ビル運用改善を行い,さらなるエネルギー消費量の削減やオーナー・利用者などの満足度向上,ビル管理者の負荷軽減を図り社会的な要請に応えるとともに,これまでの研究の学術的な発展を行いたいと考えている.



